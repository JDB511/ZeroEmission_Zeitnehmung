% 									Inhalte - Vorlage für Elektrotechnik
%                        					 2023 HTL Weiz
%----------------------------------------------------------------------------------------------------
% Dieses \LaTeX Template kann als Grundlage für die Erstellung von CPE-Dokumentationen
% verwendet werden. In diesem File werden die Inhalte in das Dokument eingebettet.
%----------------------------------------------------------------------------------------------------
\section{Einleitung}
Für die Zeitnehmung bei der Zero Emission Challenge wurde ein eigens entwickeltes System verwendet, welches in diesem Dokument beschrieben bzw. Schritt für Schritt erklärt werden soll.

\subsection{Funktionsweise}
Für die Erfassung der Rundenzeiten wurde ein spezielles System eingesetzt, das nach dem Prinzip einer Lichtschranke funktioniert. Die Funktionsweise lässt sich grundlegend wie folgt erklären.

\begin{enumerate}
	\item \textbf{Lichtschranke:} Wird die Lichtschranke beim Start- bzw. Zieltor durchfahren, wird die aktuelle Zeit gespeichert und an ein hierfür eigens entwickeltes \textbf{\ac{GUI}} für die weitere Auswertung gesendet. Um eine zuverlässige Redundanz sicherzustellen, sind jeweils zwei identische Lichtschranken-Stationen an  Start- \& Zieltor installiert.
	
	\item \textbf{\acl{GUI}:} Im \ac{GUI} werden die Zeitstempel des Start- sowie Zieltors angezeigt und können nach der Auswahl der Challenge, Team, Versuchsnummer und der umgefahrenen Hütchen in die \textbf{\ac{DB}} gespeichert werden.
	
	\item \textbf{\acl{DB}:} In der \ac{DB} werden alle Teams, Challenges, Daten der einzelnen Versuche sowie alle aufgezeichneten Zeitstempel aller Lichtschranken gespeichert.
	
	\item \textbf{Webserver:} Hier werden die individuellen Versuche der einzelnen Teams nach der jeweiligen Challenge gewertet und übersichtlich dargestellt.
\end{enumerate}






